\documentclass[a4paper,11pt]{article}
\usepackage[utf8]{inputenc}
\usepackage[french]{babel} 
\usepackage[T1]{fontenc} 
\usepackage{textcomp}
\usepackage{amsmath,amssymb}
\usepackage{mathrsfs}
\usepackage{stmaryrd}
\usepackage{graphicx}
\usepackage[titlepage,fancysections]{polytechnique}
\graphicspath{{Image/}}

\title{Microstructure des prix financiers}
\author{Antoine GROSNIT et Yassin Hamaoui}
\subtitle{MODAL - MAP474D}
\date{Juin 2018}

\begin{document}
\maketitle
\section{Une modélisation simplifiée}

\subsection{Prix négatifs}

\subsubsection{Simulation par un Monte-Carlo naïf}

$inf P<0$
On commence par un modèle simple où on simule M processus de Poisson associés à notre modèle. On détermine ensuite $P_{est}=\mathbb{P}(\inf_{t\leq T} P_{t}< 0)$ par un Monte-Carlo naïf qui consiste à utiliser l'estimateur: $P_{est}=\frac{1}{M}\sum 1_{ \inf_{t\leq T} P_{t}< 0}$ \\

Pour obtenir un intervalle de confiance, on utilise le résultat qui affirme que : 
$\sqrt{M}(\frac{1}{M}\sum 1_{ \inf_{t\leq T} P_{t}< 0}-\mathbb{P}(\inf_{t\leq T} P_{t}< 0)) \Rightarrow N(0,\mathbb{P}(\inf_{t\leq T} P_{t}< 0)(1-\mathbb{P}(\inf_{t\leq T} P_{t}< 0))$ \\

Alors un intervalle de confiance à $0.95$ est donné par : $[P_{est}-2*P_{est}(1-P_{est});P_{est}+2*P_{est}(1-P_{est})$ 


On regroupe les résultats pour différents paramètres dans le tableau suivant: \\
Tableau: $P0; i=0 ou 3 ;M=10^6; Pest; Intervalle de conf$



\subsubsection{Simulation par changement de loi}

Il s'agit maintenant d'utiliser une méthode qui permet d'évaluer correctement la probabilité quand l'évènement est rare et que le résultat donné par un Monte-Carlo naïf n'est plus pertinent (ce qui est ici le cas pour $P_{0}=35$. On va alors procéder à un changement de loi via la transformation d'Esscher. L'idée de cette technique est de modifier les probabilités de manière à rendre l'évènement étudiée moins rare. Dans ce cas, on veut que le prix diminue. Il faut donc que les sauts négatifs soient privilégiés. 

DESCRIPTION DE LA TRANSFORMATION DESSCHER

On choisit $\theta$ qui minimise la variance de l'estimation de la probabilité. Pour cela, on commence par tracer $P_{est}$ en fonction de $\theta$. On obtient le graphique suivant :

INSERER GRAPHIQUE.

On note alors un plateau dans la région A COMPLETER. On cherche dans un deuxième temps le $\theta$ de cette région qui minimise la variance de l'estimation. 
On obtient $\theta= A COMPLETER$

TABLEAU DE RESULTAT
$M1=86 * 10^6$
$M3=50 * 10^6$
$i=1, i=3$
$theta=opt, P0=10$
$theta=val opt, P0=35$

\subsubsection{Simulation par MCMC}

\subsection{Calcul du quantile}

\subsubsection{Simulation par un Monte-Carlo naïf}
Dans cette partie on veut estimer des quantiles du prix final $p_{T}$ à différents niveaux. Une première approche consiste à utiliser l'estimateur des quantiles empiriques. Il s'agit donc de simuler différentes M processus de Poisson et de réordonner les prix finaux obtenus par ordre croissant : $p_{T}^{1}, p_{T}^{2}..., p_{T}^{M}$. Le quantile empirique au niveau $\alpha$ est alors : $p_{T}^{\left \lceil M\alpha  \right \rceil}$

On obtient les résultats suivants pour différents paramètres:
$i=1 ou 3, P0=35, alpha=10^-4, 5, 6$
box plot + histogramme

\subsubsection{Simulation par changement de loi}
Dans cette partie, on veut estimer le quantile à des niveaux plus extrêmes ($10^{-5}$ ou $10^{-6}$) ce qui nécessite d'avoir recours à un changement de loi car la méthode naïve n'aboutit pas un résultat exploitable. L'idée est de nouveau utiliser la transformation d'Esscher, déterminer le paramètre $\theta$ qui minimise la variance puis effectuer le quantile en utilisant le résultat suivant du cours :
Pour $X_{1},...,X_{n}$ simulés sous $\mathbb{Q}$ : 
$Q(\alpha)=\inf_{x} \left \{ \frac{1}{n}\sum_{i=1}^{n} \frac{p(X_{i})}{q(X_{i})}1_{X_{i}\geq x}\geq \alpha \right \}$

On implémente cela en triant les prix puis en leur affectant leur poids (correspondant à la normalisation calculée précédemment).

On obtient les résultats suivants pour différents paramètres:
$i=1 ou 3, P0=35, alpha=10^-4, 5, 6$

box plot + histogramme


\section{Superposition de processus}

\subsection{Prix négatifs}

\subsubsection{Simulation par un Monte-Carlo naïf}
On procède de la même manière qu'à la section précédente en simulant M fois l'évolution du prix qui est la somme d'un processus de Poisson et d'un processus déterministe (mis à part sa première valeur) qui alterne entre saut positif et saut négatif. 

On obtient les résultats suivants:
Tableau: $P0; i=0 ou 3 ;M=10^6; Pest; Intervalle de conf$

\subsubsection{Simulation par MCMC}

\subsection{Calcul du quantile}

\subsubsection{Simulation par un Monte-Carlo naïf}
De la même qu'au calcul de quantile avec le processus précédent, on obtient les résultats suivants:

\subsubsection{Simulation par MCMC}

\section{Modélisation markovienne}

\subsection{Quelques propriétés}

\begin{itemize}

\item Signe de $\alpha_{+}$ et $\alpha_{-}$

Dans le cadre de cette modélisation on considère que $\hat{J}_{n}$ est une chaîne de Markov de matrice de transition :
$\hat{Q} = \begin{pmatrix}
\frac{1+\alpha_{+}}{2} & \frac{1-\alpha_{+}}{2} \\ 
 \frac{1-\alpha_{-}}{2}  & \frac{1+\alpha_{-}}{2} 
\end{pmatrix}$
Cela s'interprete en termes probabilistes par :
$\mathbb{P}(\hat{J}_{n+1}=1\mid \hat{J}_{n}=1)=\frac{1+\alpha_{+}}{2}$ et on veut que cette probabilité soit inférieure à $\frac{1}{2}$ pour modéliser le retour à la moyenne. On a donc : $\alpha_{+}<0$. Par le même argument: $\alpha_{-}<0$
A partir de maintenant (et sauf mention contraire) $\alpha_{+}=\alpha_{-}=\alpha$.

\item $\mathbb{P} (\hat{J}_{n}\hat{J}_{n+1}=1) \simeq \frac{1+\alpha }{2}$

On considère dans la suite que $\alpha = -0.875$ (donné par l'article de référence). On simule simplement la suite des signes $hat{J}_{n}$  en respectant la loi de transition donnée par la matrice. En utilisant la loi des grands nombres, on obtient bien:
$\mathbb{P} (\hat{J}_{n}\hat{J}_{n+1}=1) \simeq 0.0625 \simeq \frac{1+\alpha }{2}$

\end{itemize}

\subsection{Prix négatifs}


\subsubsection{Simulation par un Monte-Carlo naïf}
On va simuler la suite des prix et utiliser la loi des grands nombres pour estimer la probabilité que le prix devienne négatif. On obtient les résultats suivants:


\subsubsection{Simulation par changement de loi}


\subsection{Calcul du quantile}

\subsubsection{Simulation par un Monte-Carlo naïf}
On utilise la simulation obtenue précédemment pour déterminer le quantile grâce au quantile empirique.

\subsubsection{Simulation par changement de loi}

\section{Limite macroscopique}




\end{document}
