\documentclass[a4paper,11pt]{article}
\usepackage[utf8]{inputenc}
\usepackage[french]{babel} 
\usepackage[T1]{fontenc} 
\usepackage{textcomp}
\usepackage{amsmath,amssymb}
\usepackage{mathrsfs}
\usepackage{stmaryrd}
\usepackage{graphicx}
\usepackage[titlepage,fancysections]{polytechnique}
\graphicspath{{Image/}}

\title{Microstructure des prix financiers}
\author{Antoine GROSNIT et Yassin Hamaoui}
\subtitle{MODAL - MAP474D}
\date{Juin 2018}

\begin{document}
\maketitle
\section{Une modélisation simplifiée}

\subsection{Prix négatifs}

\subsubsection{Simulation par un Monte-Carlo naïf}

inf P<0
On commence par un modèle simple où on simule M processus de Poisson associés à notre modèle. On détermine ensuite $P_{est}=\mathbb{P}(\inf_{t\leq T} P_{t}< 0)$ par un Monte-Carlo naïf qui consiste à utiliser l'estimateur: $P_{est}=\frac{1}{M}\sum 1_{ \inf_{t\leq T} P_{t}< 0}$ \\

Pour obtenir un intervalle de confiance, on utilise le résultat qui affirme que : 
$\sqrt{M}(\frac{1}{M}\sum 1_{ \inf_{t\leq T} P_{t}< 0}-\mathbb{P}(\inf_{t\leq T} P_{t}< 0)) \Rightarrow N(0,\mathbb{P}(\inf_{t\leq T} P_{t}< 0)(1-\mathbb{P}(\inf_{t\leq T} P_{t}< 0))$ \\

Alors un intervalle de confiance à $0.95$ est donné par : $[P_{est}-2*P_{est}(1-P_{est});P_{est}+2*P_{est}(1-P_{est})$ 


On regroupe les résultats pour différents paramètres dans le tableau suivant: \\
Tableau: $P0; i=0 ou 3 ;M=10^6; Pest; Intervalle de conf$



\subsubsection{Simulation par changement de loi}

Il s'agit maintenant d'utiliser une méthode qui permet d'évaluer correctement la probabilité quand l'évènement est rare et que le résultat donné par un Monte-Carlo naïf n'est plus pertinent (ce qui est ici le cas pour $P_{0}=35$. On va alors procéder à un changement de loi via la transformation d'Esscher. L'idée de cette technique est de modifier les probabilités de manière à rendre l'évènement étudiée moins rare. Dans ce cas, on veut que le prix diminue. Il faut donc que les sauts négatifs soient privilégiés. 

DESCRIPTION DE LA TRANSFORMATION DESSCHER

On choisit $\theta$ qui minimise la variance de l'estimation de la probabilité. Pour cela, on commence par tracer $P_{est}$ en fonction de $\theta$. On obtient le graphique suivant :

INSERER GRAPHIQUE.

On note alors un plateau dans la région A COMPLETER. On cherche dans un deuxième temps le $\theta$ de cette région qui minimise la variance de l'estimation. 
On obtient $\theta= A COMPLETER$

TABLEAU DE RESULTAT


\end{document}